% \iffalse meta-comment
% !TEX program = pdfLaTeX
%<*internal>
\iffalse
%</internal>
%<*readme>
================================================================

fullwidth --- adjustbale margins
version 0.1

Released under the LaTeX Project Public License v1.3c or later
See http://www.latex-project.org/lppl.txt
----------------------------------------------------------------

The package provides the environment fullwidth with ....

================================================================
%</readme>
%<*internal>
\fi
\def\nameofplainTeX{plain}
\ifx\fmtname\nameofplainTeX\else
  \expandafter\begingroup
\fi
%</internal>
%<*install>
\input docstrip.tex
\keepsilent
\askforoverwritefalse
\preamble
----------------------------------------------------------------
fullwidth --- adjustbale margins
version 0.1

Released under the LaTeX Project Public License v1.3c or later
See http://www.latex-project.org/lppl.txt
----------------------------------------------------------------

\endpreamble
\postamble
================================================================
Copyright (C) 2011 by Marco Daniel

This work may be distributed and/or modified under the
conditions of the LaTeX Project Public License (LPPL), either
version 1.3c of this license or (at your option) any later
version.  The latest version of this license is in the file:

http://www.latex-project.org/lppl.txt

This work is "maintained" (as per LPPL maintenance status) by
Marco Daniel.

This work consists of the file  fullwidth.dtx
and the derived files           fullwidth.ins,
                                fullwidth.pdf and
                                fullwidth.sty.
================================================================
\endpostamble
\usedir{tex/latex/fullwidth}
\generate{\file{fullwidth.sty}{\from{fullwidth.dtx}{package}}}
%</install>
%<install>\endbatchfile
%<*internal>
\usedir{source/latex/fullwidth}
\generate{\file{fullwidth.ins}{\from{fullwidth.dtx}{install}}}
\nopreamble\nopostamble
\usedir{doc/latex/fullwidth}
\generate{\file{README.txt}{\from{fullwidth.dtx}{readme}}}
\ifx\fmtname\nameofplainTeX
  \expandafter\endbatchfile
\else
  \expandafter\endgroup
\fi
%</internal>
%<*driver>
\documentclass[parskip=false,english,11pt]{ltxmdf}
\usepackage{doc}
\usepackage[T1]{fontenc}
\usepackage{lmodern}
\usepackage{fullwidth}
\usepackage[numbered]{hypdoc}
\makeatletter
\renewcommand\tableofcontents{%
  \setcounter{tocdepth}{1}%
  \begin{multicols}{2}[\centering\textbf{\sffamily\Large\contentsname}]%
    \@starttoc{toc}
  \end{multicols}
}
\ifoot{}
\makeatother
\usepackage{babel}
\usepackage[utf8]{inputenc}
\usepackage[T1]{fontenc}
\usepackage{lipsum}
%\usepackage[T1,altbullet]{lucidabr}
\usepackage[scaled=0.82]{beramono}

\usepackage{fullwidth}
%\fullwidthsetup

\def\fwdname{\texttt{fullwidth}\xspace}
\introduction{This package provides the environment \fwdname which allows to set the left and right margins in a very simple way. It also allows page breaks. If you are using the \texttt{twoside} mode you can set the inner and outer margins. \par
By defining new environments the user may choose between several individual designs.%
}

\EnableCrossrefs
\CodelineIndex
\RecordChanges
\begin{document}
  \DocInput{fullwidth.dtx}
\end{document}
%</driver>
% \fi
%
% \GetFileInfo{fullwidth.sty}
%
% \DoNotIndex{\newrobustcmd,\DeclareBoolOption,\DeclareListParser,\deflength}
% \DoNotIndex{\z@,\@ne,\@M}
% \DoNotIndex{\define@key}
% \DoNotIndex{\@nil,\dimen@,\dimen@i}
% \DoNotIndex{\@mpfn,\@mpfootins,\@mpfootnotetext,\c@mpfootnote}
% \DoNotIndex{\advance,\begingroup,\ifx,\fi,\else,\@plus,\@height,\@width,\dp,\ht,\wd}
% \DoNotIndex{\closeout,\day,\def,\edef,\else,\empty,\endgroup,\csname,\endcsname,\endinput}
%
%
%
%
% \title{^^A
%   \textsf{fullwidth} --- description text\thanks{^^A
%     This file describes version \fileversion, last revised \filedate.^^A
%   }^^A
% }
% \author{^^A
%   Marco Daniel^^A
% }
% \date{Released \filedate}
%
% \maketitle
%
% \changes{v0.1}{2011/11/25}{First public release}
%
%
% \section{Motivation}
% The package is inspired by many questions on \href{http://tex.stackexchange.com/}
% {tex stackexchange}. For example in the question
% \href{http://tex.stackexchange.com/questions/34368/how-to-switch-between-two-margin-sizes}^^A
% {\emph{How to switch between two margin sizes?}} I solved the problem by using
% the package \mdname. However, since the package wasn't designed for such a 
% environment, I decided to create a new package based on the algorithm of \mdname.
%
%
%
% \section{Syntax}\label{sec:syntax}
% The package itself loads the packages \mdpack{kvoptions} and \mdpack{etoolbox}.
%
% Load the package as usual:
% \begin{mdexample}
%  \usepackage[<GLOBAL OPTIONS>]{fullwidth}
% \end{mdexample}
%
% The package defines only one environment with the following syntax:
% \begin{mdexample}
%  \begin{fullwidth}[<LOCAL OPTIONS>]
%     <CONTENT>
%  \end{fullwidth}
% \end{mdexample}
%
%
% \minisec{Autodetecting floats}
% I added a detection of \mdpack{float} and \mdpack{minipage} environments.
% If \fwdname would be used within such an environment, \fwdname it would
% automatically use the option \mdoption{nobreak}.
%
%
% %%%%%%%%%%%%%%%%%%%%%%%
% \section{Options}
% The package provides various options for manipulating frames. All options are 
% listed in the following section. Some internal macros are not shown in this
% documentation, though they can be manipulated.
% \mdDescribeCmd{fullwidthsetup}The listed option can also be set via \mdcommand{fullwidthsetup}.
%
% \subsection{Options with lengths}
%
% All lengths will be evaluated by the \eTeX-command \mdcommand{dimexpr}.
%
% I know that the predefined lengths are not well prepared. Maybe I will change it later.
%
% \mdDescribeMacro[0pt]{width} Sets the width of the whole \fwdname environment.
%
% \mdDescribeMacro[\textbackslash topskip]{skipabove}
%   Sets an additional skip above the frame.
% \mdDescribeMacro[\textbackslash topskip]{skipbelow}
%   Sets an additional skip below the frame.
%
% \mdDescribeMacro[0pt]{leftmargin}
%   Sets the length of the left margin of the environment.
% \mdDescribeMacro[0pt]{rightmargin}
%   Sets the length of the right margin of the environment.
%
% \mdDescribeMacro[.4\textbackslash baselineskip]{innertopmargin}
%   Sets the length of the inner top margin of the environment.
% \mdDescribeMacro[.4\textbackslash baselineskip]{innerbottommargin}
%   Sets the length of the inner bottom margin of the environment.
%
% \mdDescribeMacro[0pt]{outermargin}
%   Sets the length of the outer margin. This option is only avaidable in \texttt{twoside} mode.
% \mdDescribeMacro[0pt]{innermargin}
%   Sets the length of the inner margin. This option is only avaidable in \texttt{twoside} mode.
%
%
% \mdDescribeMacro[\textbackslash topsep]{splittopskip}
%   Sets the length of the skip above the split part of the environment.
% \mdDescribeMacro[0pt]{splitbottomskip}
%   Sets the length of the skip below the split part of the environment.
%
%
% \subsection{General options}\label{genopt}
%
% \mdDescribeMacro[true]{twosidemode}
%   The package detects wether \mdcommand{twoside} mode is active and uses \mdoption{innermargin}
%   and \mdoption{outermagin} by default. If you don't know this you can use this option.
%
% \mdDescribeMacro[0pt]{needspace}
%   Sometimes it is useful to set a minimum height before the environment should be splitted.
%   For such cases you can use \mdoption{needspace}. The option requires a length which sets
%   the minimum height before a frame will be splitted.
%
% \subsection{Footnotes}
% Inside the environment you can use the command \mdcommand{footnote} as usual.
%  \fwdname uses the syntax of the environment \mdpack{minipage} with the same counter.
%
% Every footnote text will be collected inside a box and will be displayed at the end of the environment \fwdname.
%
% \mdDescribeMacro[\mbox{} \mdcommand{bigskipamount}]{footnotedistance}
%   The length is the distance between the end of the environment \fwdname and
%   the displaying of the \mdcommand{footnoterule}.
%
% %\mdDescribeMacro[true]{footnoteinside}
% %  The position of the footnotes can be changed with the option \mdoption{footnoteinside}.
% %   The footnotes will be displayed at the end of the environment but you can
% %   decide whether the output is inside \fwdname or after.
% %
% %\vskip\baselineskip
% %\noindent\textbf{Note}\qquad  The output of the footnotes with the option
% %   \mdoption{footnoteinside=false} are not in a splitted frame. I think it isn't
% %   useful because the first line of a new page shouldn't be a footnote.
% 
%
% \section{Known Problems}
% In this section I will collect known problems. In case you encounter any further problems, please
% drop me an email, \href{mailto:marco.daniel@mada-nada.de}{marco.daniel at mada-nada.de}.
%
% Do you have any ideas / wishes on further extensions to this package? Please let me know!
%
% \section{Acknowledgements}
% Thanks to the members of \href{http://tex.stackexchange.com/}{tex stackexchange}.
%
% \section{Revision history}\label{rev}
% \raggedright
% \minisec{Version 0.1 submitted ???????}
% \begin{itemize*}
% \item first upload to \href{http://dante.ctan.org/upload}{CTAN}
% \end{itemize*}
% \clearpage
% \newgeometry{left=5.5cm}
% \addto\MacroFont{\color{blue!60!black}}
% \preto\theCodelineNo{\color{gray}}
% \lstMakeShortInline[basicstyle=\small\ttfamily\color{blue!60!black}]{|}
% \setlength\MacroTopsep{10pt}
%
% \section{Implementation}\label{implementation}
%
% And finally, here's how it all works\ldots
%
%\StopEventually{^^A
%  \PrintChanges
%  \PrintIndex
%}
%
%    \begin{macrocode}
%<*package>
%    \end{macrocode}
%
% \subsection{Preliminaries}
%    \begin{macrocode}
\NeedsTeXFormat{LaTeX2e}
\ProvidesPackage{fullwidth}[2011/11/25 v0.1 Provides a full-width environment]
%    \end{macrocode}
%
%    \begin{macrocode}
\RequirePackage{kvoptions,etoolbox}
\RequirePackage{zref-abspage}
\SetupKeyvalOptions{family=fwd,prefix=fwd@}
%    \end{macrocode}
%
%    \begin{macro}{\fwd@dolist}
%     Create an loop by using the package |etoolbox|. For more details
%     have a look at the documentation of the package.
%    \begin{macrocode}
\DeclareListParser*{\fwd@dolist}{,}
%    \end{macrocode}
%    \end{macro}
%
%    \begin{macro}{\fwd@do@lengthoption}
%    \begin{macro}{\fwd@lengthoption@doubledo}
%    \begin{macro}{\fdw@define@key@length}
%    \begin{macro}{\fdw@define@key@length}
%      In This part all options with length will be defined. Therefor the macro
%      |\fwd@lengthoption@doubledo| defines all length and set the default value.
%      The options will be passed via the macro |\fwd@do@lengthoption|. The macro
%      |\fdw@define@key@length| is only a short form of |define@key|.
%    \begin{macrocode}
\def\fwd@do@lengthoption#1{%
  \fwd@lengthoption@doubledo#1\@nil%
}
%    \end{macrocode}
%    \end{macro}
%
%    \begin{macrocode}
\def\fwd@lengthoption@doubledo#1==#2\@nil{%
  \csdef{fwd@#1}{}%
  \expandafter\newlength\csname fwd@#1@length\endcsname%
  \expandafter\deflength\csname fwd@#1@length\endcsname{#2\relax}%
  \fdw@define@key@length{#1}%
}
%    \end{macrocode}
%
%    \begin{macrocode}
\newrobustcmd*{\fdw@define@key@length}[1]{%
  \define@key{fwd}{#1}{%
    \expandafter\deflength\csname fwd@#1@length\endcsname{##1\relax}%
  }%
}
%    \end{macrocode}
%
%    \begin{macrocode}
\fwd@dolist{\fwd@do@lengthoption}{%
  {width==\linewidth},%
  {skipabove==\z@},%
  {skipbelow==\z@},%
  {leftmargin==\z@},%
  {rightmargin==\z@},%
  {innertopmargin==0.4\baselineskip},%
  {innerbottommargin==0.4\baselineskip},%
  {splittopskip==\z@},%
  {splitbottomskip==\topsep},%
  {outermargin==\z@},%
  {innermargin==\z@},%
  {footenotedistance==\medskipamount},
}
%    \end{macrocode}
%    \end{macro}
%    \end{macro}
%    \end{macro}
%
%    \begin{macro}{needspace}
%    \begin{macro}{twosidemode, nobreak}
%     Next the definition of the string option |needspace| and
%     the bool option will be provided.
%    \begin{macrocode}
\define@key{fwd}{needspace}[\z@]{%
  \begingroup%
    \setlength{\dimen@}{#1}%
    \vskip\z@\@plus\dimen@%
    \penalty -100\vskip\z@\@plus -\dimen@%
    \vskip\dimen@%
    \penalty 9999%
    \vskip -\dimen@%
    \vskip\z@skip % hide the previous |\vskip| from |\addvspace|
  \endgroup%
}
\DeclareBoolOption[true]{twosidemode}
\DeclareBoolOption{nobreak}
\DeclareBoolOption{footnoteinside}
%    \end{macrocode}
%    \end{macro}
%    \end{macro}
%
%    \begin{macro}{\fullwidthsetup}
%      Process all defined options and define the short form of |\setkeys{fwd}|
%    \begin{macrocode}
\ProcessKeyvalOptions*\relax
\newrobustcmd{\fullwidthsetup}{\setkeys{fwd}}
%    \end{macrocode}
%    \end{macro}
%
%    \begin{macro}{fwd@lrbox}
%      The package saved the contents in a savebox. Therefor the package
%      defines a modified |lrbox| where the lenght |\hsize|, |\columnwidth|,
%      |\textwidth| and |\linewidth| set direct.
%    \begin{macrocode}
\let\fwd@lrbox\lrbox
\patchcmd\fwd@lrbox\hbox\vbox{}{}
\patchcmd\fwd@lrbox\color@setgroup{%
  \color@setgroup%
  \hsize=\fwd@width@length%
  \columnwidth=\hsize%
  \textwidth=\hsize%
  \linewidth=\hsize%
}{}{}
\def\endfwd@lrbox{\unskip\color@endgroup}
%    \end{macrocode}
%    \end{macro}
%
%    \begin{macro}{fwd@trivlist}
%      The package defines a modified trivlist to use the internal trivlist
%      for the complete |fullwidth| environment.
%    \begin{macrocode}
\let\fwd@trivlist\trivlist
\let\endfwd@trivlist\endtrivlist
\patchcmd\endfwd@trivlist\@endparenv\fwd@endparenv{}{}
\def\fwd@endparenv{%
  \addpenalty\@endparpenalty\addvspace\fwd@skipbelow@length\@endpetrue}
%    \end{macrocode}
%    \end{macro}
%
%    \begin{macro}{\fwd@footnoterule}
%     The package uses a modified version of the footnote system of a |minipage|.
%     The macro |fwd@footnoterule| represent a small footnote line with a length of
%     2.6 pt.
%    \begin{macrocode}
\newrobustcmd*\fwd@footnoterule{%
  \kern0\p@%
  \hrule \@width 1in \kern 2.6\p@}
%    \end{macrocode}
%    \end{macro}
%
%    \begin{macro}{\fwd@footnoteoutput}
%     Equal to minipage (see \href{http://www.ctan.org/pkg/source2e}{source2e})%
%     with an added length to set the sep between text
%     and footnote. The length is set by the option |footenotedistance|.
%    \begin{macrocode}
\newrobustcmd*\fwd@footnoteoutput{%
  \ifvoid\@mpfootins\else
    \nobreak%
    \vskip\fwd@footenotedistance@length%
    \normalcolor%
    \fwd@footnoterule
    \unvbox\@mpfootins
  \fi%
}
%    \end{macrocode}
%    \end{macro}
%
%    \begin{macro}{\fwd@footnoteinput}
%     Save every footnote in a save box do save them. The comment above 
%     can be used equal. (source2e)
%    \begin{macrocode}
\newrobustcmd*\fwd@footnoteinput{%
  \def\@mpfn{mpfootnote}%
  \def\thempfn{\thempfootnote}%
  \c@mpfootnote\z@%
  \let\@footnotetext\@mpfootnotetext%
}
%    \end{macrocode}
%    \end{macro}
%
%    \begin{macro}{\fwd@detect@output}
%    \begin{macro}{\fwd@standalone}
%    \begin{macro}{\fwd@putframe}
%     To allow pagebreaks the package must check whether the package is inside
%     a non breakable environment like |minipage|, |table| or |figure|. The command
%     |\fwd@detect@output| is used at the beginning of the environment |fullwidth|.
%     If the package is used inside one of them the definition of |\fwd@standalone|
%     is adabted otherwise the definition of  |\fwd@putframe|.
%    \begin{macrocode}
\let\fwd@reserved@a\@empty
\newrobustcmd*\fwd@detect@output{%
%    \end{macrocode}
%  Test whether the option |nobreak| is set.
%    \begin{macrocode}
  \iffwd@nobreak
    \def\fwd@reserved@a{\fwd@standalone}%
  \else
    \def\fwd@reserved@a{\fwd@putframe}%
%    \end{macrocode}
%  |\@floatpenalty| is only used inside an float environment
%  ( see \href{http://tex.stackexchange.com/questions/27172/how-can-i-detect-if-im-inside-or-outside-of-a-float-environment}^^A
%        {tex.stackexchange How can I detect if I'm inside or outside of a float environment?}
%    \begin{macrocode}
    \ifnum\@floatpenalty<0\relax
      \if@twocolumn%
        \ifx\@captype\@undefined
          \def\fwd@reserved@a{\fwd@putframe}%
        \else
          \PackageWarning{fullwidth}{fullwidth inside float  ^^J
                              fullwidth uses option nobreak}%
          \def\fwd@reserved@a{\fwd@standalone}%
        \fi
      \else
        \PackageWarning{fullwidth inside float  ^^J
                             fullwidth uses option nobreak}%
        \def\fwd@reserved@a{\fwd@standalone}%
      \fi%
    \fi%
%    \end{macrocode}
%  |\if@minipage| is true inside a |minipage| environment.
%    \begin{macrocode}
    \if@minipage%
      \PackageWarning{fullwidth inside minipage  ^^J
                             fullwidth uses option nobreak}%
      \def\fwd@reserved@a{\fwd@standalone}%
    \fi%
%    \end{macrocode}
%  |\ifinner| test for an internal mode (see texbook)
%    \begin{macrocode}
    \ifinner%
      \PackageWarning{fullwidth inside a box ^^J
                            fullwidth uses option nobreak }%
      \def\fwd@reserved@a{\fwd@standalone}%
    \fi%
  \fi%
  \fwd@reserved@a%
}
%    \end{macrocode}
%    \end{macro}
%    \end{macro}
%    \end{macro}
%
%    \begin{macrocode}
\def\fullwidth{\@ifnextchar[\fullwidth@i\fullwidth@ii}%
\def\fullwidth@ii{\fullwidth@i[]}%
\def\fullwidth@i[#1]{% default-Umgebung
  \begingroup
  \fullwidthsetup{#1}%%
  \fwd@twoside@checklength%
  \let\width\z@%
  \let\height\z@%
  \setlength{\topsep}{\fwd@skipabove@length}%
  \begingroup%
    \let\partopsep\z@%
    \expandafter\endgroup%
  \begin{fwd@trivlist}\item\relax%
  \hsize=\fwd@width@length\relax%
  \fwd@footnoteinput%
  \begin{fwd@lrbox}{\@tempboxa}%
 }
%    \end{macrocode}
%
%    \begin{macrocode}
\def\endfullwidth{%
  \iffwd@footnoteinside%
    \def\fwd@reserveda{%
      \fwd@footnoteoutput%
      \end{fwd@lrbox}%
      \let\hsize\linewidth
      \fwd@detect@output%
    }%
  \else%
    \def\fwd@reserveda{%
      \end{fwd@lrbox}
      \let\hsize\linewidth
      \fwd@detect@output%
      \fwd@footnoteoutput%
    }%
  \fi%
  \fwd@reserveda%
  \end{fwd@trivlist}%
  \hrule \@height\z@ \@width\hsize
  \endgroup%\@endparenv%
}
%    \end{macrocode}
%
% Two-side mode
%    \begin{macrocode}
\newrobustcmd*\fwd@twoside@checklength{%
  \if@twoside
    \setlength\fwd@rightmargin@length{\fwd@outermargin@length}%
    \setlength\fwd@leftmargin@length{\fwd@innermargin@length}%
  \else
    \boolfalse{fwd@twosidemode}%
  \fi%
}
%    \end{macrocode}
%
%    \begin{macrocode}
\newcounter{fwd@zref@counter}%keine doppelten laebes
\zref@newprop*{fwd@pagevalue}[0]{\number\value{page}}
\zref@addprop{\ZREF@mainlist}{fwd@pagevalue}
%    \end{macrocode}
%
%    \begin{macrocode}
\newrobustcmd*\fwd@zref@label{%
  \stepcounter{fwd@zref@counter}%
  \zref@label{fwd@pagelabel-\number\value{fwd@zref@counter}}%
}
%    \end{macrocode}
%
%    \begin{macrocode}
\newrobustcmd*\if@fwd@pageodd{%
  \zref@refused{fwd@pagelabel-\the\value{fwd@zref@counter}}%
  \ifodd\zref@extract{fwd@pagelabel-\the\value{fwd@zref@counter}}{fwd@pagevalue}%
    \edef\fwd@reserveda{\fwd@pageisodd}%
  \else
    \edef\fwd@reserveda{\fwd@pageiseven}%
  \fi
  \fwd@reserveda%
}
%    \end{macrocode}
%
%    \begin{macrocode}
\newrobustcmd*\fwd@pageisodd{%
  \setlength\fwd@rightmargin@length{\fwd@outermargin@length}%
  \setlength\fwd@leftmargin@length{\fwd@innermargin@length}%
}
%    \end{macrocode}
%
%    \begin{macrocode}
\newrobustcmd*\fwd@pageiseven{%
  \setlength\fwd@leftmargin@length{\fwd@outermargin@length}%
  \setlength\fwd@rightmargin@length{\fwd@innermargin@length}%
}
%    \end{macrocode}
%
%    \begin{macrocode}
\newrobustcmd*\fwd@@setzref{\fwd@zref@label\if@fwd@pageodd}
%    \end{macrocode}
%
%    \begin{macrocode}
%%==================================================%%
%%================= Platz auf Seite ================%%
%%==================================================%%
\newlength\fwd@freevspace@length
\newrobustcmd*\fwd@freepagevspace{%
  \penalty\@M \vskip 2\baselineskip \vskip\height
  \penalty9999 \vskip -2\baselineskip \vskip-\height
  \penalty9999
  \ifdimequal{\pagegoal}{\maxdimen}%
    {\fwd@freevspace@length\vsize}%
    {\fwd@freevspace@length=\pagegoal\relax%
      \advance\fwd@freevspace@length by -\pagetotal\relax%
    }%
}
%    \end{macrocode}
%
%    \begin{macrocode}
\newrobustcmd*\fwd@standalone{\relax%
  \ifvoid\@tempboxa\relax
    \PackageWarning{fullwidth}{The environment is empty\MessageBreak}%
    \let\fwd@reserved@a\relax%
  \else
    \def\fwd@reserved@a{\fwd@putbox@tempboxa}%
  \fi
  \fwd@reserved@a%
}
%    \end{macrocode}
%
%    \begin{macrocode}
\def\fwd@putframe{\relax%
  \ifvoid\@tempboxa\relax
    \PackageWarning{fullwidth}{The environment is empty\MessageBreak}%
    \let\fwd@reserved@a\relax%
  \else
    \fwd@freepagevspace%
    \ifdimless{\fwd@freevspace@length}{2\baselineskip}{%
      \PackageInfo{fullwidth}{Not enough space on this page}%die Seite hat nur noch minimal Platz
      \vfill\eject%
      \def\fwd@reserved@a{\fwd@putframe}%
    }{%
      \ifdimless{\ht\@tempboxa+\dp\@tempboxa}{\fwd@freevspace@length}{%passt auf Seite%
        \begingroup
        \ifbool{fwd@twosidemode}{\fwd@@setzref}{}%
          \fwd@putbox@tempboxa%%
          \endgroup
          \let\fwd@reserved@a\relax%
      }{%
        \def\fwd@reserved@a{\fwd@putframe@i}%
      }%passt nicht auf Seite
    }%
  \fi
  \fwd@reserved@a%
}
%    \end{macrocode}
%
%    \begin{macrocode}
\def\fwd@putframe@i{%Box muss gesplittet werden -- Ausgabe der ersten Teilbox
  %Berechnung der Splittgroesse Abstand oben
  \fwd@freepagevspace%
  \dimen@=\the\fwd@freevspace@length%
  \dimen@i=\fwd@innertopmargin@length%
  \advance\dimen@i by 2\baselineskip%
  \ifdim\dimen@<\dimen@i\relax
    \hrule \@height\z@ \@width\hsize%
    \vfill\eject%
    \def\fwd@reserved@a{\fwd@putframe}%
  \else%
    \ifdimless{\ht\@tempboxa+\dp\@tempboxa}{\dimen@}{%
      \PackageWarning{fullwidth}{You got a bad break\MessageBreak
                                you have to change it manually\MessageBreak
                                by changing the text, the space\MessageBreak
                                or something else}%
      \advance\dimen@ by -1.8\baselineskip\relax%
    }{}%
    \advance\dimen@ by -1pt\relax%Box darf nicht zu Groß werden.
    \splitmaxdepth\z@ \splittopskip\fwd@splittopskip@length%
    \setbox\tw@\vsplit\@tempboxa to \dimen@
    \setbox\tw@\vbox{\unvbox\tw@}%needed?
    \ifdimgreater{\ht\tw@+\dp\tw@}{\dimen@}{%Falsch gesplittet
      \PackageInfo{fullwidth}{Box was splittet wrong\MessageBreak}%
      \dimen@i=\dimen@
      \advance\dimen@ by -\ht\tw@
      \advance\dimen@ by -\dp\tw@
      \advance\dimen@i by 0.5\dimen@
      \splittopskip\z@%
      \setbox\@tempboxa\vbox{\unvbox\tw@%
                                \hrule \@height\dp\strutbox \@width\z@%benoetigt um Tiefe zu haben
                                \unvbox\@tempboxa}
      \splittopskip\fwd@splittopskip@length%
      \setbox\tw@\vsplit\@tempboxa to \dimen@i
      \setbox\tw@\vbox{\unvbox\tw@}%
    }{}%
    \setbox\@tempboxa\vbox{\unvbox\@tempboxa}%PRUEFEN!!!!
    \ifvoid\@tempboxa
      \PackageWarning{fullwidth}{You got a bad break\MessageBreak
                          because the splittet box is empty\MessageBreak
                          You have to change the page settings\MessageBreak
                          like enlargethispage or something else}%
      \setbox\@tempboxa\vbox{\box\tw@\box\@tempboxa}%
      \def\fwd@reserved@a{\fwd@putframe}%
    \fi
    \ifvoid\tw@%%pruefe, ob erste Box leer ist
      \hrule \@height\z@ \@width\hsize
      \vfill\eject%
      \def\fwd@reserved@a{\fwd@putframe}%
    \else
      \ifdimequal{\ht\tw@}{0pt}{%
        \hrule \@height\z@ \@width\hsize%
        \vfill\eject%
        \setbox\@tempboxa\vbox{\unvbox\tw@\unvbox\@tempboxa}
        \def\fwd@reserved@a{\fwd@putframe}%
      }{%
        \begingroup
          \ifbool{fwd@twosidemode}{\fwd@@setzref}{}%
            \fwd@putbox@tw@%Groesse des Splittens passt
        \endgroup
        \hrule \@height\z@ \@width\hsize
        \vfill\eject%
        \def\fwd@reserved@a{\fwd@putframe@ii}%
      }%
    \fi%
  \fi%
  \fwd@reserved@a%
}
%    \end{macrocode}
%
%    \begin{macrocode}
\def\fwd@putframe@ii{%Ausgabe der mittleren Box(en) wenn vorhanden
  \setlength{\fwd@freevspace@length}{\vsize}%
  \setlength{\dimen@}{\dimexpr\ht\@tempboxa+\dp\@tempboxa\relax}%
  \ifdimgreater{\dimen@}{\fwd@freevspace@length}{%
    \advance\fwd@freevspace@length by -\fwd@splitbottomskip@length
    \splitmaxdepth\z@ \splittopskip\fwd@splittopskip@length%
    \setbox\tw@\vsplit\@tempboxa to \fwd@freevspace@length%
    \setbox\tw@\vbox{\unvbox\tw@}%PRUEFEN!!!
    \setbox\@tempboxa\vbox{\unvbox\@tempboxa}%PRUEFEN!!!!
    \ifvoid\@tempboxa\relax%
      \fwd@PackageWarning{You got a bad break\MessageBreak
                          because the split box is empty\MessageBreak
                          You have to change the settings}%
    \fi%
    \begingroup
      \ifbool{fwd@twosidemode}{\fwd@@setzref}{}%
      \fwd@putbox@tw@%
    \endgroup
    \hrule \@height\z@ \@width\hsize
    \vfill\eject
    \def\fwd@reserved@a{\fwd@putframe@ii}%
  }{%Hier die Ausgabe der mittleren Box
    \ifvoid\@tempboxa
      \fwd@PackageWarning{You got a bad break\MessageBreak
                          because the last split box is empty\MessageBreak
                          You have to change the settings}%
    \fi%
    \begingroup
      \ifbool{fwd@twosidemode}{\fwd@@setzref}{}%
        \fwd@putbox@tempboxa%
    \endgroup
    \let\fwd@reserved@a\relax%
  }%Hier kommt die Ausgabe der letzten Box
  \fwd@reserved@a%
}
%    \end{macrocode}
%
%    \begin{macrocode}
\newrobustcmd\fwd@leftline[1]{\leftline{\hspace*{\fwd@leftmargin@length}#1}}%Keine Modifikation
%    \end{macrocode}
%
%    \begin{macrocode}
\newrobustcmd\fwd@putbox@tempboxa{\fwd@leftline{\box\@tempboxa}}
\newrobustcmd\fwd@putbox@tw@{\fwd@leftline{\box\tw@}}
%    \end{macrocode}
%
%    \begin{macrocode}
\endinput
}
%    \end{macrocode}
%
%    \begin{macrocode}
%</package>
%    \end{macrocode}
%\Finale
